\documentclass[../../main.tex]{subfiles}
\begin{document}

\subsection*{10.12}
Una bobina quadrata, di lato $a = 2\ cm$ e resistenza $R = 0.1\ \Omega$, disposta con due lati verticali, ruota con velocità angolare costante $\omega$ attorno all'asse verticale passante per il centro. Essa è immersa in un campo magnetico $B =0.6\ T$ uniforme e costante, ortogonale all'asse di rotazione, ed è elimentata da un generatore di resistenza interna nulla che ornisce la f.e.m. $\varepsilon = 0.2 + 0.24sin(\omega t)\ V$. Si osserva che durante il moto la corrente $i$ nella bobina resta costante.\\
Calcoalre la corrente $i$ e la velocità angolare $\omega$.
\subsubsection*{Formule utilizzate}
\subsubsection*{Soluzione}
La f.e.m. indotta è data dalla variazione del flusso di campo magnetico causato dalla rotazione della bobina:\\
$\varepsilon_i = -\frac{d\Phi(B)}{dt} = -B\Sigma\frac{d\alpha}{dt} = -B\Sigma \omega sin(\omega t)$\\
Imponendo che questa espressione annulli la componente dipendente dal tempo del generatore $[0.24 sin(\omega t)]$ si ottiene: $\omega = 10^3\ \frac{rad}{s}$\\
Inoltre la corrente circolante sarà data da: $i = \frac{\varepsilon(t) + \varepsilon_i}{R} = \frac{0.2}{R} = 2\ A$
\newpage

\end{document}