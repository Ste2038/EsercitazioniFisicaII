\documentclass[../../main.tex]{subfiles}
\begin{document}

\subsection*{8.8}
Una sottile striscia metallica di larghezza $h = 2\ cm$ è percorsa dalla corrente $i = 10\ A$.\\
Calcolare il valore del campo magnetico $\vec{B}(x)$ a distanza x dal bordo della striscia (in particolare $x\gg h$) e il momento meccanico $\vec{M}$ che agisce su un piccolo ago magnetico di momento $\vec{m} = 0.1\vec{u_x}\ Am^2$ posto a distanza $x = 1\ cm$
\subsubsection*{Formule utilizzate}
\subsubsection*{Soluzione punto a}
\subsubsection*{Soluzione punto b}
\newpage

\end{document}