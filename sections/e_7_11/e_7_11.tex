\documentclass[../../main.tex]{subfiles}
\begin{document}

\subsection*{7.11}
Una spira quadrata di lato $a = 5\ cm$ è percorsa da corrente i.
\\Il momento magnetico della spira è $m = m_x\vec{u_x}+B_z\vec{u_z}$ con $ B_x = 0.25\ T$, $B_z=0.30\ T$.
\\Calcolare il valore della corrente i, il modulo del momento meccanico $\vec{H}$, l'angolo $\alpha$ tra $\vec{m} e \vec{B}$, l'energia potenziale magnetica $U_p$.
\subsubsection*{Formule utilizzate}
\subsubsection*{Soluzione punto a}
$m = \sqrt{m_x^2+m_y^2}= 10^{-3}\ Am^2$
\\$i = \frac{m}{a^2} = 0.4\ A$ dato $i = \frac{m}{s} con s= a^2$
\\$B = \sqrt{B_x^2+B_z^2}=0.39\ T$
\\$\vec{M} = \vec{m}\wedge\vec{B} = m_yB_z\vec{u_x}-m_xB_z\vec{u_y}-m_yB_x\vec{u_z}$
\\$M = \sqrt{\left(m_x^2+m_y^2\right)B_z^2+m_y^2B_x^2}$
\\$M = mB\sin\alpha$   $\sin\alpha = \frac{M}{mB}$
\\$U_p = -\vec{m}\wedge\vec{B} = -m_xB_x$
\subsubsection*{Soluzione punto b}
\newpage

\end{document}