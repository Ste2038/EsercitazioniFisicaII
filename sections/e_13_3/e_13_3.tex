\documentclass[../../main.tex]{subfiles}
\begin{document}
\subsection*{13.3}
Un trasmettitore emette onde elettromagnetiche in un cono che copre un angolo solido $\Delta \Omega = 2* 10^{-2} sr$.\\
A distanza $r_1 = 2\ km$ dal trasmettitore l'ampiezza massima del campo elettrico è $E_1 = 20\ \frac{V}{m}$.\\
Calcolare l'ampiezza $B_1$ del campo magnetico.\\
Calcolare la potenza P del trasfmettitore.\\
Calcolare le ampiezze $E_2$ e $B_2$ alla distanza $r_2 = 10\ km$.\\
\subsubsection*{Formule utilizzate}
\subsubsection*{Soluzione punto a}
La superficie del cono alla distanza r è $s = r^2 \Delta \Omega$.\\
L'ampiezza del campo magnetico: $B_1 = \frac{E_1}{c} = 6.67 * 10^{-8}\ T$.\\
\subsubsection*{Soluzione punto b}
La potenza P del trasmettitore: $I_1 = \frac{E_1^2}{2Z_0} = 0.52\ \frac{W}{m^2}$\\
$P = I_1 r_1^2\ \Delta\Omega = 42.4\ kW$\\
\subsubsection*{Soluzione punto c}
Possiamo trovarle sapendo che le onde sferiche decrescono linearmente.\\
$E_2 = E_1\frac{r_1}{r_2} = 4\ \frac{V}{m}$\\
$B_2 = \frac{E_2}{c} = 1.33 * 10^{-8}\ T$\\
\newpage
\end{document}