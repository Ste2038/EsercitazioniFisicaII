\documentclass[../../main.tex]{subfiles}
\begin{document}

\subsection*{3.9}
Dimostrare che la funzione $V(x, y) = ax^2+bxy-ay^2$ con a e b costanti può rappresentare una funzione potenziale.
\\Determinare il campo elettrostatico e la densità di carica $\rho(x,y)$.
\subsubsection*{Formule utilizzate}
\subsubsection*{Soluzione punto a}
Il campo si calcola mediante la relazione $E = -\nabla v$
\\$E_x = -\frac{\delta v}{\delta x} = -(2ax+by)$
\\$E_y = -\frac{\delta v}{\delta y} = -(bx-2ay)$
\\$E_z = -\frac{\delta v}{\delta z} = 0$
\\Il campo così calcolato rappresenta effettivamente un campo elettrostatico infatti soddisfa la relazione: $rot\vec{E} = \nabla \wedge  E = 0$.
\\$\frac{\delta E_z}{\delta y} - \frac{\delta E_y}{\delta z} = 0$
\\$\frac{\delta E_x}{\delta z} - \frac{\delta E_z}{\delta x} = 0$
\\$\frac{\delta E_y}{\delta x} - \frac{\delta E_x}{\delta y} = 0$
\\La densità di carica si calcola mediante il teorema di Gauss in locale $div\vec{E} = \nabla \vec{E} = \frac{\rho}{\epsilon_0}$.
\\$\frac{\delta E_x}{\delta x}+\frac{\delta E_y}{\delta y}+\frac{\delta E_z}{\delta z} = \frac{\rho}{\epsilon_0}$
\\$\rho = \epsilon_0\left(\frac{\delta E_x}{\delta x} + \frac{\delta E_y}{\delta y} + \frac{\delta E_z}{\delta z}\right) = \epsilon_0\left(-2a+2a\right) = 0$
\subsubsection*{Soluzione punto b}
\newpage

\end{document}