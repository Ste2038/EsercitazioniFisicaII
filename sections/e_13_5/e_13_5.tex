\documentclass[../../main.tex]{subfiles}
\begin{document}
\subsection*{13.5}
Un' antena parabolica ha un'apertura di 15 m e riceve in direzione normale alla sua superficie un segnale radio proveniente da una sorgente molto lontana, di ampiezza $E_0 = 4 * 10^{-7}\ \frac{V}{m}$.\\
Assumendo che l'antenna assorba tutta la radiazione che la colpisce calcolare la forza esercitata dall'onda sull'antenna.
\subsubsection*{Formule utilizzate}
\subsubsection*{Soluzione punto a}
Calcolo l'intensità:\\
$I = \frac{E_0^2}{2Z_0} = 2.1 * 10^{-16}\ \frac{W}{m^2}$\\
La potenza assorbita dall'antenna (perfettamente assorbente):\\
$P_{ass} = \frac{I}{c}\pi r^2 = 3.75 * 10^{-14}\ W$
La forza assorbita:\\
$F = p_{rad}\Sigma = \frac{I}{c}\pi r^2 = 1.25 * 10^{-22}\ N$
\newpage
\end{document}