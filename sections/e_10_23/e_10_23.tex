\documentclass[../../main.tex]{subfiles}
\begin{document}

\subsection*{10.23}
Le due armature piane circolari di un condensatore hanno area $\Sigma = 0.1m^2$, distano d e sono collegate ad un generatore di f.e.m. $V=V_0\sin\omega t$ con $V_0 = 200V$, $\Omega = 100\frac{rad}{s}$. La corrente massima di conduzione nei fili è $i_0 = 8.86\mu A$.\\
Calcolare i valori massimi della corrente di spostamento e della densità di corrente di spostamento.\\
Calcolare i valori massimi di $\frac{d\Phi(\vec{E})}{dt}$\\
Calcolare la distanza d tra le armature.\\
Calcolare il valore massimo del campo magnetico in funzione della distanza r dal centro del condensatore.
\subsubsection*{Formule utilizzate}
\subsubsection*{Soluzione punto a}
$i_{s, max} = i_{c, max}=i_0 = 8.86 * 10^{-6}\ A$\\
\subsubsection*{Soluzione punto b}
Il valore massimo di $\frac{d\Phi(\vec{E})}{dt}$, si avrà in corrispondenza della corrente di conduzione massima.\\
Dividendo per l'area $\Sigma$ otteniamo la densità di corrente.\\
$j_{s, max} = \frac{i_{s, max}}{\Sigma} = 8.86 * 10^{-5}\ \frac{A}{m^2}$\\
$\left[\frac{d\Phi(\vec{E})}{dt}\right]_{max} = \frac{i_{s, max}}{\epsilon_0} = 10^6\ \frac{Vm}{s}$\\
$i_s = \frac{dq}{dt} = \frac{d(CV)}{dt} = C\omega V_0 cos\omega t$\\
\subsubsection*{Soluzione punto c}
Possiamo ricavare la distanza fra le due armature:\\
$d = \frac{\epsilon_0 \Sigma}{i_{s ,max}}\omega V_0 = 2 mm$\\
\subsubsection*{Soluzione punto d}
Il raggio R delle armature circolari di area $\Sigma$ vale:\\
$R = \sqrt{\frac{\Sigma}{\pi}} = 0.178\ m$\\
Quindi internamente alle armature ($0 \le r \le R$):\\
$2\pi r B(r) = \mu_0 i_s \frac{\pi r^2}{\pi R^2}$\tab $B(r) = \frac{\mu_0 i_s}{2\pi R^2}$\\
$B_{max} = \frac{\mu_0 i_{s, max} r}{2\pi R^2} = 5.59 * 10^{-11} r T$\\
Esternamente alle armature ($ R \le r$):\\
$2\pi rB(r) = \mu_0 i_s \rightarrow B(r) = \frac{\mu_0 i_s}{2\pi r}$\\
$B_{max}  =\frac{\mu_0 i_{s,max}}{2\pi r} = \frac{1.77 * 10^{-12}}{r}\ T$.
\newpage
\end{document}