\documentclass[../../main.tex]{subfiles}
\begin{document}

\subsection*{8.2}
Due fili indefiniti distanti 2a=4cm, paralleli all'asse x solo percorsi indicati in figura.
\\Calcolare il campo magnetico $\vec{B}(z)$ sull'asse dei due fili e a quale distanza dal centro O si arresti un piccolo magnete lanciato con velocità $v_0 = 7.1 * 10^{-2}\frac{m}{s}$ da O lungo l'asse z, di massa $mg = 3.97 * 10^{-2}kg$ e momento magnetico $m=0.2Am^2$ parallelo e concorde a B.
\\Si assume che l'asse z orizzontale.
\subsubsection*{Formule utilizzate}
\subsubsection*{Soluzione punto a}
\subsubsection*{Soluzione punto b}
\newpage

\end{document}