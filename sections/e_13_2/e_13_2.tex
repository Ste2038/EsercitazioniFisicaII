\documentclass[../../main.tex]{subfiles}
\begin{document}
\subsection*{13.2}
Una lampada da 500W irradia tale potenza isotropicamente con efficenza del 80\%.\\
Calcolare alla distanza $r=5\ m$ l'intensità I.\\
Calcolare alla stessa distanza i valori di massimi $E_0$ del campo elettrico e $B_0$ del campo magnetico.\\
Calcolare alla stessa distanza la forza esercitata su un dischetto di raggio $a=5\ cm$, perfettamente riflettente ortogonale alla direzione di propagazione delle onde.
\subsubsection*{Formule utilizzate}
\subsubsection*{Soluzione punto a}
La potenza reale è l'80\% del totale.\\
$P_{reale} = 0.8 * 500 = 400\ W$\\
Il vettore di poynting indica il trasferimento di energia\\
$\vec{S} = \frac{\vec{E}\wedge\vec{B}}{\mu}$\\
$I = \frac{P_{erogata}}{4\pi r^2} = 1.27\ \frac{W}{m^2}$\\
\subsubsection*{Soluzione punto b}
Il valore massimo $E_0$ del campo elettrico:\\
$I = \frac{1}{2}\epsilon_0\ c\ E_0^2 = \frac{1}{2}\frac{1}{Z_0}E_0^2$\\
Poichè $\epsilon_0 c = \epsilon_0\frac{1}{\sqrt{\epsilon_0\mu_0}} = \sqrt{\frac{\epsilon_0}{\mu_0}} = \frac{1}{Z_0}$\
Quindi: $E_0 = \sqrt{2\ Z_0\ I} = 30.9\ \frac{V}{m}$\\
Il valore massimo di $B_0$ del campo magnetico è:\\
$B_0 = \frac{E_0}{c} = 10.3 * 10^{-8}\ T$
\subsubsection*{Soluzione punto c}
La forza F esercitata su un dischetto di raggio $a = 5\ cm = 0.05\ m$, perfettamente riflettente ortogonale alla direzione di propagazione delle onde si trova usando la pressione di radiazione per una superficie perfettamente riflettente:\\\\
Nel caso in cui la superficie fosse perfettametnte assorbente:\\
$p_{rad} = \frac{I}{c}$\\
Nel nostro caso invece la pressione è doppia perchè viene riflessa\\
$p_{rad} = \frac{2I}{c}$\\
$F = p_{rad} A_{disco} = \frac{2I}{c} \pi a^2 = 6.65 * 10^{-11}\ N$\\
\newpage
\end{document}