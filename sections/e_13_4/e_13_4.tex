\documentclass[../../main.tex]{subfiles}
\begin{document}
\subsection*{13.4}
Una sorgente puntiforme di microonde produce impulsi di frequenza $f = 20\ GHz$ e di durata $t = 1$ ns. La sorgente è posta nel fuoco di un paraboloide conduttore di apertura $2R = 20$ cm così che si ottiene in uscita un fascio parallelo all'asse del paraboloide. La potenza media di ogni impulso è $P = 25$ kW.\\
Calcoolare la $\lambda$ delle microonde.\\
Calcolare l'energia totale di ciascun impulso.\\
Calcolare l'intensità del fascio di microonde.\\
Calcolare la densità media di energia di ciascun impulso.\\
Calcolare l'ampiezza del campo $\vec{E}$ e del campo $\vec{B}$.\\
Calcolare la forza esercitatata durante un impulso su una superficie perfettamente riflettente ortogonale al fascio.\\
\subsubsection*{Formule utilizzate}
\subsubsection*{Soluzione punto a}
La lunghezza d'onda delle microonde:\\
$\lambda = \frac{c}{f} = 1.5\ cm$.
\subsubsection*{Soluzione punto b}
L'energia totale di ciascun impulso:\\
$\Delta U = P t = 2.5 * 10^{-5}\ J$
\subsubsection*{Soluzione punto c}
L'intensità del fascio di microonde:\\
$I = \frac{P}{\pi R^2} = 7.96 * 10^5\ \frac{W}{m^2}$
\subsubsection*{Soluzione punto d}
La densità media di energia di ciascun impulso:\\
$<u>_T = \frac{I}{c} = 2.65 * 10^{-3}\ \frac{J}{m^3}$
\subsubsection*{Soluzione punto e}
L'ampiezza del campo elettrico e del campo magnetico del fascio:\\
$E_0 = \sqrt{2 Z_0 I} = 2.45 * 10^4\ \frac{V}{m}$\\
$B_0 = \frac{E_0}{c} = 8.17 * 10^{-5}\ T$\\
\subsubsection*{Soluzione punto f}
$p_{rad} = \frac{2I}{c}$\\
$F = p_{rad} \pi R^2 = \frac{2I}{c} \pi R^2 = 1.67 * 10^{-4}\ N$
\newpage
\end{document}