\documentclass[../../main.tex]{subfiles}
\begin{document}
\subsection*{13.9}
Un granello di polvere cosmica nel sistema solare si trova soggetto parte del sole si aalla forza di attrazione gravitazionale che alla forza dovuta alla pressione di radiazione. Suppoendo che la particella sia sferica ed in grado di assorbire tutta la radiazione.\\
Calcolare il valore minimo $a_0$ del raggio al di sotto del quale la particella sarebbe spinta fuori dal sistema solare.\\
I valori numerici sono: massa del sole $M = 2 * 10^30\ kg$, potenza solare $P ? 3.96 * 10^{26}\ W$, densità del granello $\rho ? 2.7 * 10^3\ \frac{kg}{m^3}$.
\subsubsection*{Formule utilizzate}
\subsubsection*{Soluzione punto a}
Detta r la distanza del Sole:\\
$F_{rad} = p_{rad} \pi a^2 = \frac{I}{c}\pi a^2 = \frac{P}{4\pi r^2 c}\pi a^2.$.\\
$F_{grav} = \gamma\frac{M_{sole}M_{granello}}{r^2} = \gamma \frac{M_{sole}}{r^2}\rho\frac{4}{3}\pi a^3$.\\
Se la particelle sfugge al sole allora: $F_{rad} \ge F_{grav}$.\\
$a \le a_0 = \frac{3P}{16\pi \gamma c \rho M_{sole}} = 0.22\ \mu m$
\newpage
\end{document}