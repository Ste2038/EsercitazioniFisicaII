\documentclass[../../main.tex]{subfiles}
\begin{document}
\subsection*{13.8}
Una navicella spaziale di massa $m = 1.5 * 10^3\ kg$ viaggia nel vuoto in assenza di campo gravitazionale e utilizza un fascio laser di potenza $P = 10kW$ montanto in coda come propulsore.\\
Calcolare di quanto aumenta la velocità della navicella in ogni giorno di navigazione
\subsubsection*{Formule utilizzate}
\subsubsection*{Soluzione punto a}
Il raggio laser trasporta quantità di moto nella direzione di propagazione dle laser. Dalla terza legge di Newton viene fornita una quantità uguale di quantità di moto al veicolo spaziale nella direzione opposta.\\
Teorema dell'impulso: $m\Delta v = F\Delta t$.\\
Per un giorno $\Delta t = 24 * 3600\ s$\\
$F = F_{rad} = p_{rad}\Sigma = \frac{P}{c}$\\
Sostituendo si trova che:\\
$\Delta v = \frac{F\Delta t}{m} = \frac{P\Delta t}{mc} = 1.92 * 10^-3\ \frac{m}{s}$.

\newpage
\end{document}