\documentclass[../../main.tex]{subfiles}
\begin{document}

\subsection*{8.12}
Un conduttore cilindrico molto lungo di raggio $a = 2\ cm$ ha nel suo interno una cavità cilindrica di raggio $b = 0.3\ cm$, essa pure molto lunga. Gli assi dei due cilindri sono paralleli e distano $d = 1\ cm$. Nel conduttore fluisce una corrente $i = 20\ A$, distribuita uniformemente.\\
Dimostrare che il campo magnetico $\vec{B}$ all'interno della cavità è costante, calcolandone modulo e direzione.\\
Calcolare inoltre l'energia magnetica e l'induttanza per unità di lunghezza del conduttore.\\
\includegraphics[scale=0.3]{e_8_12_0.png}\\
\includegraphics[scale=0.3]{e_8_12_1.png}
\subsubsection*{Formule utilizzate}
\subsubsection*{Soluzione punto a}
\subsubsection*{Soluzione punto b}
\newpage

\end{document}